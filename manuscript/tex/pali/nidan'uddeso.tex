\section{Nidān'uddeso}
\label{nidan'uddeso}

Suṇātu me bhante / āvuso saṅgho, ajj'uposatho pannaraso / cātuddaso / sāmaggo, yadi saṅghassa pattakallaṁ, saṅgho uposathaṁ kareyya pātimokkhaṁ uddiseyya.

Kiṁ saṅghassa pubbakiccaṁ?

Pārisuddhiṁ āyasmanto ārocetha, pātimokkhaṁ uddisissāmi, taṁ sabb'eva santā sādhukaṁ suṇoma manasi karoma. Yassa siyā āpatti, so āvikareyya, asantiyā āpattiyā tuṇhī bhavitabbaṁ, tuṇhībhāvena kho panāyasmante “parisuddhā”ti vedissāmi.

Yathā kho pana paccekapuṭṭhassa veyyākaraṇaṁ hoti, evam'evaṁ evarūpāya parisāya yāvatatiyaṁ anusāvitaṁ hoti. Yo pana bhikkhu yāvatatiyaṁ anusāviyamāne saramāno santiṁ āpattiṁ n'āvikareyya, sampajānamusāvādassa hoti. Sampajānamusāvādo kho pan'āyasmanto antarāyiko dhammo vutto bhagavatā, tasmā saramānena bhikkhunā āpannena visuddh'āpekkhena santī āpatti āvikātabbā, āvikatā hi'ssa phāsu hoti.

\medskip

\begin{center}
Uddiṭṭhaṃ kho āyasmanto nidānaṃ.

\smallskip

Tatth'āyasmante pucchāmi: Kacci'ttha parisuddhā?\\
Dutiyam'pi pucchāmi: Kacci'ttha parisuddhā?\\
Tatiyam'pi pucchāmi: Kacci'ttha parisuddhā?

\smallskip

Parisuddh'etth'āyasmanto, tasmā tuṇhī, evam'etaṁ dhārayāmīti.
\end{center}

\begin{outro}
  Nidānaṃ niṭṭhitaṃ
\end{outro}

\clearpage
