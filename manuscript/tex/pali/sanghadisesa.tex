\setsecheadstyle{\sectionFmt}
\section{Saṅghādises'uddeso}
\label{sd}

\begin{intro}
  Ime kho pan'āyasmanto terasa saṅghādisesā dhammā uddesaṁ āgacchanti.
\end{intro}

\pdfbookmark[2]{Saṅghādisesa 1}{sd1}
\subsection*{\hyperref[comm1]{Saṅghādisesa 1: Sukkavissaṭṭhisikkhāpadaṁ}}
\label{sd1}
Sañcetanikā sukkavissaṭṭhi aññatra supinantā saṅghādiseso.

\pdfbookmark[2]{Saṅghādisesa 2}{sd2}
\subsection*{\hyperref[comm2]{Saṅghādisesa 2: Kāyasaṁsaggasikkhāpadaṁ}}
\label{sd2}
Yo pana bhikkhu otiṇṇo vipariṇatena cittena mātugāmena saddhiṁ kāyasaṁsaggaṁ samāpajjeyya hatthaggāhaṁ vā veṇiggāhaṁ vā aññatarassa vā aññatarassa vā aṅgassa parāmasanaṁ, saṅghādiseso.

\pdfbookmark[2]{Saṅghādisesa 3}{sd3}
\subsection*{\hyperref[comm3]{Saṅghādisesa 3: Duṭṭhullavācāsikkhāpadaṁ}}
\label{sd3}
Yo pana bhikkhu otiṇṇo vipariṇatena cittena mātugāmaṁ duṭṭhullāhi vācāhi obhāseyya yathā taṁ yuvā yuvatiṁ methunupasaṁhitāhi, saṅghādiseso.

\pdfbookmark[2]{Saṅghādisesa 4}{sd4}
\subsection*{\hyperref[comm4]{Saṅghādisesa 4: Attakāmapāricariyasikkhāpadaṁ}}
\label{sd4}
Yo pana bhikkhu otiṇṇo vipariṇatena cittena mātugāmassa santike attakāmapāricariyāya vaṇṇaṁ bhāseyya “etadaggaṁ bhagini pāricariyānaṁ yā mādisaṁ sīlavantaṁ kalyāṇadhammaṁ brahmacāriṁ etena dhammena paricareyyā”ti methunupasaṁhitena, saṅghādiseso.

\pdfbookmark[2]{Saṅghādisesa 5}{sd5}
\subsection*{\hyperref[comm5]{Saṅghādisesa 5: Sañcarittasikkhāpadaṁ}}
\label{sd5}
Yo pana bhikkhu sañcarittaṁ samāpajjeyya itthiyā vā purisamatiṁ purisassa vā itthimatiṁ, jāyattane vā jārattane vā, antamaso taṅkhaṇikāyapi, saṅghādiseso.

\pdfbookmark[2]{Saṅghādisesa 6}{sd6}
\subsection*{\hyperref[comm6]{Saṅghādisesa 6: Kuṭikārasikkhāpadaṁ}}
\label{sd6}
Saññācikāya pana bhikkhunā kuṭiṁ kārayamānena assāmikaṁ attuddesaṁ pamāṇikā kāretabbā, tatridaṁ pamāṇaṁ, dīghaso dvādasa vidatthiyo sugatavidatthiyā, tiriyaṁ sattantarā, bhikkhū abhinetabbā vatthudesanāya, tehi bhikkhūhi vatthu desetabbaṁ anārambhaṁ saparikkamanaṁ. Sārambhe ce bhikkhu vatthusmiṁ aparikkamane saññācikāya kuṭiṁ kāreyya, bhikkhū vā anabhineyya vatthudesanāya, pamāṇaṁ vā atikkāmeyya, saṅghādiseso.

\pdfbookmark[2]{Saṅghādisesa 7}{sd7}
\subsection*{\hyperref[comm7]{Saṅghādisesa 7: Vihārakārasikkhāpadaṁ}}
\label{sd7}
Mahallakaṁ pana bhikkhunā vihāraṁ kārayamānena sassāmikaṁ attuddesaṁ bhikkhū abhinetabbā vatthudesanāya, tehi bhikkhūhi vatthu desetabbaṁ anārambhaṁ saparikkamanaṁ. Sārambhe ce bhikkhu vatthusmiṁ aparikkamane mahallakaṁ vihāraṁ kāreyya, bhikkhū vā anabhineyya vatthudesanāya, saṅghādiseso.

\pdfbookmark[2]{Saṅghādisesa 8}{sd8}
\subsection*{\hyperref[comm8]{Saṅghādisesa 8: Duṭṭhadosasikkhāpadaṁ}}
\label{sd8}
Yo pana bhikkhu bhikkhuṁ duṭṭho doso appatīto amūlakena pārājikena dhammena anuddhaṁseyya “appeva nāma naṁ imamhā brahmacariyā cāveyyan''ti, tato aparena samayena samanuggāhīyamāno vā asamanuggāhīyamāno vā amūlakañceva taṁ adhikaraṇaṁ hoti, bhikkhu ca dosaṁ patiṭṭhāti, saṅghādiseso.

\pdfbookmark[2]{Saṅghādisesa 9}{sd9}
\subsection*{\hyperref[comm9]{Saṅghādisesa 9: Aññabhāgiyasikkhāpadaṁ}}
\label{sd9}
Yo pana bhikkhu bhikkhuṁ duṭṭho doso appatīto aññabhāgiyassa adhikaraṇassa kiñcidesaṁ lesamattaṁ upādāya pārājikena dhammena anuddhaṁseyya “appeva nāma naṁ imamhā brahmacariyā cāveyyan''ti, tato aparena samayena samanuggāhīyamāno vā asamanuggāhīyamāno vā aññabhāgiyañceva taṁ adhikaraṇaṁ hoti kocideso lesamatto upādinno, bhikkhu ca dosaṁ patiṭṭhāti, saṅghādiseso.

\pdfbookmark[2]{Saṅghādisesa 10}{sd10}
\subsection*{\hyperref[comm10]{Saṅghādisesa 10: Saṅghabhedasikkhāpadaṁ}}
\label{sd10}
Yo pana bhikkhu samaggassa saṅghassa bhedāya parakkameyya, bhedanasaṁvattanikaṁ vā adhikaraṇaṁ samādāya paggayha tiṭṭheyya, so bhikkhu bhikkhūhi evamassa vacanīyo “māyasmā samaggassa saṅghassa bhedāya parakkami, bhedanasaṁvattanikaṁ vā adhikaraṇaṁ samādāya paggayha aṭṭhāsi, sametāyasmā saṅghena, samaggo hi saṅgho sammodamāno avivadamāno ekuddeso phāsu viharatī”ti, evañca so bhikkhu bhikkhūhi vuccamāno tatheva paggaṇheyya, so bhikkhu bhikkhūhi yāvatatiyaṁ samanubhāsitabbo tassa paṭinissaggāya, yāvatatiyañce samanubhāsiyamāno taṁ paṭinissajjeyya, iccetaṁ kusalaṁ, no ce paṭinissajjeyya, saṅghādiseso.

\pdfbookmark[2]{Saṅghādisesa 11}{sd11}
\subsection*{\hyperref[comm11]{Saṅghādisesa 11: Bhed'ānuvattakasikkhāpadaṁ}}
\label{sd11}
Tasseva kho pana bhikkhussa bhikkhū honti anuvattakā vaggavādakā eko vā dve vā tayo vā, te evaṁ vadeyyuṁ “māyasmanto etaṁ bhikkhuṁ kiñci avacuttha, dhammavādī ceso bhikkhu, vinayavādī ceso bhikkhu, amhākañceso bhikkhu chandañca ruciñca ādāya voharati, jānāti, no bhāsati, amhākampetaṁ khamatī”ti, te bhikkhū bhikkhūhi evamassu vacanīyā “māyasmanto evaṁ avacuttha, na ceso bhikkhu dhammavādī, na ceso bhikkhu vinayavādī, māyasmantānampi saṅghabhedo ruccittha, sametāyasmantānaṁ saṅghena, samaggo hi saṅgho sammodamāno avivadamāno ekuddeso phāsu viharatī”ti, evañca te bhikkhū bhikkhūhi vuccamānā tatheva paggaṇheyyuṁ, te bhikkhū bhikkhūhi yāvatatiyaṁ samanubhāsitabbā tassa paṭinissaggāya, yāvatatiyañce samanubhāsiyamānā taṁ paṭinissajjeyyuṁ, iccetaṁ kusalaṁ, no ce paṭinissajjeyyuṁ, saṅghādiseso.

\pdfbookmark[2]{Saṅghādisesa 12}{sd12}
\subsection*{\hyperref[comm12]{Saṅghādisesa 12: Dubbacasikkhāpadaṁ}}
\label{sd12}
Bhikkhu paneva dubbacajātiko hoti uddesapariyāpannesu sikkhāpadesu bhikkhūhi sahadhammikaṁ vuccamāno attānaṁ avacanīyaṁ karoti “mā maṁ āyasmanto kiñci avacuttha kalyāṇaṁ vā pāpakaṁ vā, ahampāyasmante na kiñci vakkhāmi kalyāṇaṁ vā pāpakaṁ vā, viramathāyasmanto mama vacanāyā”ti, so bhikkhu bhikkhūhi evamassa vacanīyo “māyasmā attānaṁ avacanīyaṁ akāsi, vacanīyamevāyasmā attānaṁ karotu, āyasmāpi bhikkhū vadatu sahadhammena, bhikkhūpi āyasmantaṁ vakkhanti sahadhammena, evaṁ saṁvaddhā hi tassa bhagavato parisā yadidaṁ aññamaññavacanena aññamaññavuṭṭhāpanenā”ti, evañca so bhikkhu bhikkhūhi vuccamāno tatheva paggaṇheyya, so bhikkhu bhikkhūhi yāvatatiyaṁ samanubhāsitabbo tassa paṭinissaggāya, yāvatatiyañce samanubhāsiyamāno taṁ paṭinissajjeyya, iccetaṁ kusalaṁ, no ce paṭinissajjeyya, saṅghādiseso.

\pdfbookmark[2]{Saṅghādisesa 13}{sd13}
\subsection*{\hyperref[comm13]{Saṅghādisesa 13: Kuladūsakasikkhāpadaṁ}}
\label{sd13}
Bhikkhu paneva aññataraṁ gāmaṁ vā nigamaṁ vā upanissāya viharati kuladūsako pāpasamācāro, tassa kho pāpakā samācārā dissanti ceva suyyanti ca, kulāni ca tena duṭṭhāni dissanti ceva suyyanti ca, so bhikkhu bhikkhūhi evamassa vacanīyo “āyasmā kho kuladūsako pāpasamācāro, āyasmato kho pāpakā samācārā dissanti ceva suyyanti ca, kulāni cāyasmatā duṭṭhāni dissanti ceva suyyanti ca, pakkamatāyasmā imamhā āvāsā, alaṁ te idha vāsenā”ti, evañca so bhikkhu bhikkhūhi vuccamāno te bhikkhū evaṁ vadeyya “chandagāmino ca bhikkhū, dosagāmino ca bhikkhū, mohagāmino ca bhikkhū, bhayagāmino ca bhikkhū tādisikāya āpattiyā ekaccaṁ pabbājenti, ekaccaṁ na pabbājentī”ti, so bhikkhu bhikkhūhi evamassa vacanīyo “māyasmā evaṁ avaca, na ca bhikkhū chandagāmino, na ca bhikkhū dosagāmino, na ca bhikkhū mohagāmino, na ca bhikkhū bhayagāmino, āyasmā kho kuladūsako pāpasamācāro, āyasmato kho pāpakā samācārā dissanti ceva suyyanti ca, kulāni cāyasmatā duṭṭhāni dissanti ceva suyyanti ca, pakkamatāyasmā imamhā āvāsā, alaṁ te idha vāsenā”ti, evañca so bhikkhu bhikkhūhi vuccamāno tatheva paggaṇheyya, so bhikkhu bhikkhūhi yāvatatiyaṁ samanubhāsitabbo tassa paṭinissaggāya, yāvatatiyañce samanubhāsiyamāno taṁ paṭinissajjeyya, iccetaṁ kusalaṁ, no ce paṭinissajjeyya, saṅghādiseso.

\medskip

\begin{center}
Uddiṭṭhā kho āyasmanto terasa saṅghādisesā dhammā nava paṭhamāpattikā, cattāro yāvatatiyakā. Yesaṁ bhikkhu aññataraṁ vā aññataraṁ vā āpajjitvā yāvatīhaṁ jānaṁ paṭicchādeti, tāvatīhaṁ tena bhikkhunā akāmā parivatthabbaṁ. Parivutthaparivāsena bhikkhunā uttari chārattaṁ bhikkhumānattāya paṭipajjitabbaṁ, ciṇṇamānatto bhikkhu yattha siyā vīsatigaṇo bhikkhusaṅgho, tattha so bhikkhu abbhetabbo. Ekenapi ce ūno vīsatigaṇo bhikkhusaṅgho taṁ bhikkhuṁ abbheyya, so ca bhikkhu anabbhito, te ca bhikkhū gārayhā, ayaṁ tattha sāmīci.

\smallskip

Tatth'āyasmante pucchāmi: Kacci'ttha parisuddhā?\\
Dutiyam'pi pucchāmi: Kacci'ttha parisuddhā?\\
Tatiyam'pi pucchāmi: Kacci'ttha parisuddhā?

\smallskip

Parisuddh'etth'āyasmanto, tasmā tuṇhī, evam'etaṁ dhārayāmī.
\end{center}

\begin{outro}
  Saṅghādiseso niṭṭhito
\end{outro}

\clearpage
