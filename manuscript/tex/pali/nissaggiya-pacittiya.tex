\section{Nissaggiyapācittiyā}
\label{np}

\begin{intro}
  Ime kho pan'āyasmanto tiṁsa nissaggiyā pācittiyā dhammā uddesaṁ āgacchanti.
\end{intro}

\setsubsecheadstyle{\subsectionFmt}
\subsection{Cīvaravaggo}
\vspace{0.2cm}

\pdfbookmark[3]{Nissaggiya Pācittiya 1}{np1}
\subsubsection*{\hyperref[forf-exp1]{Nissaggiya Pācittiya 1: Kaṭhinasikkhāpadaṁ}}
\label{np1}
Niṭṭhitacīvarasmiṁ bhikkhunā ubbhatasmiṁ kathine das'āhaparamaṁ atirekacīvaraṁ dhāretabbaṁ, taṁ atikkāmayato nissaggiyaṁ pācittiyaṁ.

\pdfbookmark[3]{Nissaggiya Pācittiya 2}{np2}
\subsubsection*{\hyperref[forf-exp2]{Nissaggiya Pācittiya 2: Uddositasikkhāpadaṁ}}
\label{np2}
Niṭṭhitacīvarasmiṁ bhikkhunā ubbhatasmiṁ kathine ekarattam'pi ce bhikkhu ticīvarena vippavaseyya, aññatra bhikkhusammutiyā nissaggiyaṁ pācittiyaṁ.

\pdfbookmark[3]{Nissaggiya Pācittiya 3}{np3}
\subsubsection*{\hyperref[forf-exp3]{Nissaggiya Pācittiya 3: Akālacīvarasikkhāpadaṁ}}
\label{np3}
Niṭṭhitacīvarasmiṁ bhikkhunā ubbhatasmiṁ kathine bhikkhuno pan'eva akālacīvaraṁ uppajjeyya, ākaṅkhamānena bhikkhunā paṭiggahetabbaṁ, paṭiggahetvā khippam'eva kāretabbaṁ, no c'assa pāripūri, māsaparamaṁ tena bhikkhunā taṁ cīvaraṁ nikkhipitabbaṁ ūnassa pāripūriyā satiyā paccāsāya. Tato ce uttari nikkhipeyya satiyā'pi paccāsāya, nissaggiyaṁ pācittiyaṁ.

\pdfbookmark[3]{Nissaggiya Pācittiya 4}{np4}
\subsubsection*{\hyperref[forf-exp4]{Nissaggiya Pācittiya 4: Purāṇacīvarasikkhāpadaṁ}}
\label{np4}
Yo pana bhikkhu aññātikāya bhikkhuniyā purāṇacīvaraṁ dhovāpeyya vā rajāpeyya vā ākoṭāpeyya vā, nissaggiyaṁ pācittiyaṁ.

\pdfbookmark[3]{Nissaggiya Pācittiya 5}{np5}
\subsubsection*{\hyperref[forf-exp5]{Nissaggiya Pācittiya 5: Cīvarappaṭiggahaṇasikkhāpadaṁ}}
\label{np5}
Yo pana bhikkhu aññātikāya bhikkhuniyā hatthato cīvaraṁ paṭiggaṇheyya aññatra pārivattakā, nissaggiyaṁ pācittiyaṁ.

\pdfbookmark[3]{Nissaggiya Pācittiya 6}{np6}
\subsubsection*{\hyperref[forf-exp6]{Nissaggiya Pācittiya 6: Aññātakaviññattisikkhāpadaṁ}}
\label{np6}
Yo pana bhikkhu aññātakaṁ gahapatiṁ vā gahapatāniṁ vā cīvaraṁ viññāpeyya aññatra samayā, nissaggiyaṁ pācittiyaṁ. Tatth'āyaṁ samayo, acchinnacīvaro vā hoti bhikkhu, naṭṭhacīvaro vā, ayaṁ tattha samayo.

\pdfbookmark[3]{Nissaggiya Pācittiya 7}{np7}
\subsubsection*{\hyperref[forf-exp7]{Nissaggiya Pācittiya 7: Tat'uttarisikkhāpadaṁ}}
\label{np7}
Tañ'ce aññātako gahapati vā gahapatānī vā bahūhi cīvarehi abhihaṭṭhuṁ pavāreyya, santar'uttaraparamaṁ tena bhikkhunā tato cīvaraṁ sāditabbaṁ. Tato ce uttari sādiyeyya, nissaggiyaṁ pācittiyaṁ.

\pdfbookmark[3]{Nissaggiya Pācittiya 8}{np8}
\subsubsection*{\hyperref[forf-exp8]{Nissaggiya Pācittiya 8: Paṭhama-upakkhaṭasikkhāpadaṁ}}
\label{np8}
Bhikkhuṁ pan'eva uddissa aññātakassa gahapatissa vā gahapatāniyā vā cīvaracetāpannaṁ upakkhaṭaṁ hoti “iminā cīvaracetāpannena cīvaraṁ cetāpetvā itthan'nāmaṁ bhikkhuṁ cīvarena acchādessāmī”ti, tatra ce so bhikkhu pubbe appavārito upasaṅkamitvā cīvare vikappaṁ āpajjeyya “sādhu vata maṁ āyasmā iminā cīvaracetāpannena evarūpaṁ vā evarūpaṁ vā cīvaraṁ cetāpetvā acchādehī”ti kalyāṇakamyataṁ upādāya, nissaggiyaṁ pācittiyaṁ.

\pdfbookmark[3]{Nissaggiya Pācittiya 9}{np9}
\subsubsection*{\hyperref[forf-exp9]{Nissaggiya Pācittiya 9: Dutiya-upakkhaṭasikkhāpadaṁ}}
\label{np9}
Bhikkhuṁ pan'eva uddissa ubhinnaṁ aññātakānaṁ gahapatīnaṁ vā gahapatānīnaṁ vā paccekacīvaracetāpannāni upakkhaṭāni honti “imehi mayaṁ paccekacīvaracetāpannehi paccekacīvarāni cetāpetvā itthan'nāmaṁ bhikkhuṁ cīvarehi acchādessāmā”ti, tatra ce so bhikkhu pubbe appavārito upasaṅkamitvā cīvare vikappaṁ āpajjeyya “sādhu vata maṁ āyasmanto imehi paccekacīvaracetāpannehi evarūpaṁ vā evarūpaṁ vā cīvaraṁ cetāpetvā acchādetha ubho'va santā ekenā”ti kalyāṇakamyataṁ upādāya, nissaggiyaṁ pācittiyaṁ.

\pdfbookmark[3]{Nissaggiya Pācittiya 10}{np10}
\subsubsection*{\hyperref[forf-exp10]{Nissaggiya Pācittiya 10: Rājasikkhāpadaṁ}}
\label{np10}
Bhikkhuṁ pan'eva uddissa rājā vā rājabhoggo vā brāhmaṇo vā gahapatiko vā dūtena cīvaracetāpannaṁ pahiṇeyya “iminā cīvaracetāpannena cīvaraṁ cetāpetvā itthan'nāmaṁ bhikkhuṁ cīvarena acchādehī”ti. So ce dūto taṁ bhikkhuṁ upasaṅkamitvā evaṁ vadeyya “idaṁ kho, bhante, āyasmantaṁ uddissa cīvaracetāpannaṁ ābhataṁ, paṭiggaṇhātu āyasmā cīvaracetāpannan''ti. Tena bhikkhunā so dūto evam'assa vacanīyo “na kho mayaṁ, āvuso, cīvaracetāpannaṁ paṭiggaṇhāma, cīvarañ'ca kho mayaṁ paṭiggaṇhāma kālena kappiyan''ti. So ce dūto taṁ bhikkhuṁ evaṁ vadeyya “atthi pan'āyasmato koci veyyāvaccakaro”ti. Cīvar'atthikena, bhikkhave, bhikkhunā veyyāvaccakaro niddisitabbo ārāmiko vā upāsako vā “eso kho, āvuso, bhikkhūnaṁ veyyāvaccakaro”ti. So ce dūto taṁ veyyāvaccakaraṁ saññāpetvā taṁ bhikkhuṁ upasaṅkamitvā evaṁ vadeyya “yaṁ kho, bhante, āyasmā veyyāvaccakaraṁ niddisi, saññatto so mayā, upasaṅkamat'āyasmā kālena, cīvarena taṁ acchādessatī”ti. Cīvar'atthikena, bhikkhave, bhikkhunā veyyāvaccakaro upasaṅkamitvā dvattikkhattuṁ codetabbo sāretabbo “attho me, āvuso, cīvarenā”ti, dvattikkhattuṁ codayamāno sārayamāno taṁ cīvaraṁ abhinipphādeyya, icc'etaṁ kusalaṁ, no ce abhinipphādeyya, catukkhattuṁ pañcakkhattuṁ chakkhattuparamaṁ tuṇhībhūtena uddissa ṭhātabbaṁ, catukkhattuṁ pañcakkhattuṁ chakkhattuparamaṁ tuṇhībhūto uddissa tiṭṭhamāno taṁ cīvaraṁ abhinipphādeyya, icc'etaṁ kusalaṁ, tato ce uttari vāyamamāno taṁ cīvaraṁ abhinipphādeyya, nissaggiyaṁ pācittiyaṁ. No ce abhinipphādeyya, yatassa cīvaracetāpannaṁ ābhataṁ, tattha sāmaṁ vā gantabbaṁ, dūto vā pāhetabbo “yaṁ kho tumhe āyasmanto bhikkhuṁ uddissa cīvaracetāpannaṁ pahiṇittha, na taṁ tassa bhikkhuno kiñci atthaṁ anubhoti, yuñjant'āyasmanto sakaṁ, mā vo sakaṁ vinassā”ti, ayaṁ tattha sāmīci.

\begin{center}
  Kathinavaggo paṭhamo.
\end{center}

\subsection{Eḷakalomavaggo}
\vspace{0.2cm}

\pdfbookmark[3]{Nissaggiya Pācittiya 11}{np11}
\subsubsection*{\hyperref[forf-exp11]{Nissaggiya Pācittiya 11: Kosiyasikkhāpadaṁ}}
\label{np11}
Yo pana bhikkhu kosiyamissakaṁ santhataṁ kārāpeyya, nissaggiyaṁ pācittiyaṁ.

\pdfbookmark[3]{Nissaggiya Pācittiya 12}{np12}
\subsubsection*{\hyperref[forf-exp12]{Nissaggiya Pācittiya 12: Suddhakāḷakasikkhāpadaṁ}}
\label{np12}
Yo pana bhikkhu suddhakāḷakānaṁ eḷakalomānaṁ santhataṁ kārāpeyya, nissaggiyaṁ pācittiyaṁ.

\pdfbookmark[3]{Nissaggiya Pācittiya 13}{np13}
\subsubsection*{\hyperref[forf-exp13]{Nissaggiya Pācittiya 13: Dvebhāgasikkhāpadaṁ}}
\label{np13}
Navaṁ pana bhikkhunā santhataṁ kārayamānena dve bhāgā suddhakāḷakānaṁ eḷakalomānaṁ ādātabbā, tatiyaṁ odātānaṁ, catutthaṁ gocariyānaṁ. Anādā ce bhikkhu dve bhāge suddhakāḷakānaṁ eḷakalomānaṁ, tatiyaṁ odātānaṁ, catutthaṁ gocariyānaṁ, navaṁ santhataṁ kārāpeyya, nissaggiyaṁ pācittiyaṁ.

\pdfbookmark[3]{Nissaggiya Pācittiya 14}{np14}
\subsubsection*{\hyperref[forf-exp14]{Nissaggiya Pācittiya 14: Chabbassasikkhāpadaṁ}}
\label{np14}
Navaṁ pana bhikkhunā santhataṁ kārāpetvā chabbassāni dhāretabbaṁ, orena ce channaṁ vassānaṁ taṁ santhataṁ vissajjetvā vā avissajjetvā vā aññaṁ navaṁ santhataṁ kārāpeyya aññatra bhikkhusammutiyā, nissaggiyaṁ pācittiyaṁ.

\pdfbookmark[3]{Nissaggiya Pācittiya 15}{np15}
\subsubsection*{\hyperref[forf-exp15]{Nissaggiya Pācittiya 15: Nisīdanasanthatasikkhāpadaṁ}}
\label{np15}
Nisīdanasanthataṁ pana bhikkhunā kārayamānena purāṇasanthatassa sāmantā sugatavidatthi ādātabbā dubbaṇṇakaraṇāya. Anādā ce bhikkhu purāṇasanthatassa sāmantā sugatavidatthiṁ, navaṁ nisīdanasanthataṁ kārāpeyya, nissaggiyaṁ pācittiyaṁ.

\pdfbookmark[3]{Nissaggiya Pācittiya 16}{np16}
\subsubsection*{\hyperref[forf-exp16]{Nissaggiya Pācittiya 16: Eḷakalomasikkhāpadaṁ}}
\label{np16}
Bhikkhuno pan'eva addhānamaggappaṭipannassa eḷakalomāni uppajjeyyuṁ, ākaṅkhamānena bhikkhunā paṭiggahetabbāni, paṭiggahetvā tiyojanaparamaṁ sahatthā haritabbāni asante hārake. Tato ce uttari hareyya, asantepi hārake, nissaggiyaṁ pācittiyaṁ.

\pdfbookmark[3]{Nissaggiya Pācittiya 17}{np17}
\subsubsection*{\hyperref[forf-exp17]{Nissaggiya Pācittiya 17: Eḷakalomadhovāpanasikkhāpadaṁ}}
\label{np17}
Yo pana bhikkhu aññātikāya bhikkhuniyā eḷakalomāni dhovāpeyya vā rajāpeyya vā vijaṭāpeyya vā, nissaggiyaṁ pācittiyaṁ.

\pdfbookmark[3]{Nissaggiya Pācittiya 18}{np18}
\subsubsection*{\hyperref[forf-exp18]{Nissaggiya Pācittiya 18: Rūpiyasikkhāpadaṁ}}
\label{np18}
Yo pana bhikkhu jātarūparajataṁ uggaṇheyya vā uggaṇhāpeyya vā upanikkhittaṁ vā sādiyeyya, nissaggiyaṁ pācittiyaṁ.

\pdfbookmark[3]{Nissaggiya Pācittiya 19}{np19}
\subsubsection*{\hyperref[forf-exp19]{Nissaggiya Pācittiya 19: Rūpiyasaṁvohārasikkhāpadaṁ}}
\label{np19}
Yo pana bhikkhu nānappakārakaṁ rūpiyasaṁvohāraṁ samāpajjeyya, nissaggiyaṁ pācittiyaṁ.

\pdfbookmark[3]{Nissaggiya Pācittiya 20}{np20}
\subsubsection*{\hyperref[forf-exp20]{Nissaggiya Pācittiya 20: Kayavikkayasikkhāpadaṁ}}
\label{np20}
Yo pana bhikkhu nānappakārakaṁ kayavikkayaṁ samāpajjeyya, nissaggiyaṁ pācittiyaṁ.

\begin{center}
  Kosiyavaggo dutiyo
\end{center}

\subsection{Pattavaggo}
\vspace{0.2cm}

\pdfbookmark[3]{Nissaggiya Pācittiya 21}{np21}
\subsubsection*{\hyperref[forf-exp21]{Nissaggiya Pācittiya 21: Pattasikkhāpadaṁ}}
\label{np21}
Das'āhaparamaṁ atirekapatto dhāretabbo, taṁ atikkāmayato nissaggiyaṁ pācittiyaṁ.

\pdfbookmark[3]{Nissaggiya Pācittiya 22}{np22}
\subsubsection*{\hyperref[forf-exp22]{Nissaggiya Pācittiya 22: Ūnapañcabandhanasikkhāpadaṁ}}
\label{np22}
Yo pana bhikkhu ūnapañcabandhanena pattena aññaṁ navaṁ pattaṁ cetāpeyya, nissaggiyaṁ pācittiyaṁ. Tena bhikkhunā so patto bhikkhuparisāya nissajjitabbo, yo ca tassā bhikkhuparisāya pattapariyanto, so tassa bhikkhuno padātabbo “ayaṁ te bhikkhu patto yāva bhedanāya dhāretabbo”ti, ayaṁ tattha sāmīci.

\pdfbookmark[3]{Nissaggiya Pācittiya 23}{np23}
\subsubsection*{\hyperref[forf-exp23]{Nissaggiya Pācittiya 23: Bhesajjasikkhāpadaṁ}}
\label{np23}
Yāni kho pana tāni gilānānaṁ bhikkhūnaṁ paṭisāyanīyāni bhesajjāni, seyyathidaṁ—sappi navanītaṁ telaṁ madhu phāṇitaṁ, tāni paṭiggahetvā satt'āhaparamaṁ sannidhikārakaṁ paribhuñjitabbāni, taṁ atikkāmayato nissaggiyaṁ pācittiyaṁ.

\pdfbookmark[3]{Nissaggiya Pācittiya 24}{np24}
\subsubsection*{\hyperref[forf-exp24]{Nissaggiya Pācittiya 24: Vassikasāṭikasikkhāpadaṁ}}
\label{np24}
“Māso seso gimhānan''ti bhikkhunā vassikasāṭikacīvaraṁ pariyesitabbaṁ, “addhamāso seso gimhānan''ti katvā nivāsetabbaṁ. Orena ce “māso seso gimhānan''ti vassikasāṭikacīvaraṁ pariyeseyya, orena“ddhamāso seso gimhānan''ti katvā nivāseyya, nissaggiyaṁ pācittiyaṁ.

\pdfbookmark[3]{Nissaggiya Pācittiya 25}{np25}
\subsubsection*{\hyperref[forf-exp25]{Nissaggiya Pācittiya 25: Cīvara-acchindanasikkhāpadaṁ}}
\label{np25}
Yo pana bhikkhu bhikkhussa sāmaṁ cīvaraṁ datvā kupito anattamano acchindeyya vā acchindāpeyya vā, nissaggiyaṁ pācittiyaṁ.

\pdfbookmark[3]{Nissaggiya Pācittiya 26}{np26}
\subsubsection*{\hyperref[forf-exp26]{Nissaggiya Pācittiya 26: Suttaviññattisikkhāpadaṁ}}
\label{np26}
Yo pana bhikkhu sāmaṁ suttaṁ viññāpetvā tantavāyehi cīvaraṁ vāyāpeyya, nissaggiyaṁ pācittiyaṁ.

\pdfbookmark[3]{Nissaggiya Pācittiya 27}{np27}
\subsubsection*{\hyperref[forf-exp27]{Nissaggiya Pācittiya 27: Mahāpesakārasikkhāpadaṁ}}
\label{np27}
Bhikkhuṁ pan'eva uddissa aññātako gahapati vā gahapatānī vā tantavāyehi cīvaraṁ vāyāpeyya, tatra ce so bhikkhu pubbe appavārito tantavāye upasaṅkamitvā cīvare vikappaṁ āpajjeyya “idaṁ kho, āvuso, cīvaraṁ maṁ uddissa viyyati, āyatañ'ca karotha, vitthatañca, appitañca, suvītañca, suppavāyitañca, suvilekhitañca, suvitacchitañ'ca karotha, app'eva nāma mayam'pi āyasmantānaṁ kiñcimattaṁ anupadajjeyyāmā”ti. Evañ'ca so bhikkhu vatvā kiñcimattaṁ anupadajjeyya antamaso piṇḍapātamattampi, nissaggiyaṁ pācittiyaṁ.

\pdfbookmark[3]{Nissaggiya Pācittiya 28}{np28}
\subsubsection*{\hyperref[forf-exp28]{Nissaggiya Pācittiya 28: Accekacīvarasikkhāpadaṁ}}
\label{np28}
Dasāhānāgataṁ kattikatemāsikapuṇṇamaṁ bhikkhuno pan'eva accekacīvaraṁ uppajjeyya, accekaṁ maññamānena bhikkhunā paṭiggahetabbaṁ, paṭiggahetvā yāva cīvarakālasamayaṁ nikkhipitabbaṁ. Tato ce uttari nikkhipeyya, nissaggiyaṁ pācittiyaṁ.

\pdfbookmark[3]{Nissaggiya Pācittiya 29}{np29}
\subsubsection*{\hyperref[forf-exp29]{Nissaggiya Pācittiya 29: Sāsaṅkasikkhāpadaṁ}}
\label{np29}
Upavassaṁ kho pana kattikapuṇṇamaṁ yāni kho pana tāni āraññakāni senāsanāni sāsaṅkasammatāni sappaṭibhayāni, tathārūpesu bhikkhu senāsanesu viharanto ākaṅkhamāno tiṇṇaṁ cīvarānaṁ aññataraṁ cīvaraṁ antaraghare nikkhipeyya, siyā ca tassa bhikkhuno kocideva paccayo tena cīvarena vippavāsāya, chārattaparamaṁ tena bhikkhunā tena cīvarena vippavasitabbaṁ. Tato ce uttari vippavaseyya aññatra bhikkhusammutiyā, nissaggiyaṁ pācittiyaṁ.

\pdfbookmark[3]{Nissaggiya Pācittiya 30}{np30}
\subsubsection*{\hyperref[forf-exp30]{Nissaggiya Pācittiya 30: Pariṇatasikkhāpadaṁ}}
\label{np30}
Yo pana bhikkhu jānaṁ saṅghikaṁ lābhaṁ pariṇataṁ attano pariṇāmeyya, nissaggiyaṁ pācittiyaṁ.

\begin{center}
  Pattavaggo tatiyo
\end{center}

\medskip

\begin{center}
Uddiṭṭhā kho āyasmanto tiṁsa nissaggiyā pācittiyā dhammā.

\smallskip

Tatth'āyasmante pucchāmi: Kacci'ttha parisuddhā?\\
Dutiyam'pi pucchāmi: Kacci'ttha parisuddhā?\\
Tatiyam'pi pucchāmi: Kacci'ttha parisuddhā?

\smallskip

Parisuddh'etth'āyasmanto, tasmā tuṇhī, evam'etaṁ dhārayāmi.
\end{center}

\begin{outro}
  Nissaggiyapācittiyā niṭṭhitā
\end{outro}

\clearpage
