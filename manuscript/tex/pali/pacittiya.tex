\section{Pācittiyā}
\label{pc}

\begin{intro}
  Ime kho pan'āyasmanto dvenavuti pācittiyā dhammā uddesaṁ āgacchanti.
\end{intro}

\subsection{Musāvādavaggo}
\vspace{0.2cm}

\pdfbookmark[3]{Pācittiya 1}{pac1}
\subsubsection*{\hyperref[exp1]{Pācittiya 1: Musāvādasikkhāpadaṁ}}
\label{pac1}

Sampajānamusāvāde pācittiyaṁ.

\pdfbookmark[3]{Pācittiya 2}{pac2}
\subsubsection*{\hyperref[exp2]{Pācittiya 2: Omasavādasikkhāpadaṁ}}
\label{pac2}

Omasavāde pācittiyaṁ.

\pdfbookmark[3]{Pācittiya 3}{pac3}
\subsubsection*{\hyperref[exp3]{Pācittiya 3: Pesuññasikkhāpadaṁ}}
\label{pac3}

Bhikkhupesuññe pācittiyaṁ.

\pdfbookmark[3]{Pācittiya 4}{exp4}
\subsubsection*{\hyperref[exp4]{Pācittiya 4: Padasodhammasikkhāpadaṁ}}
\label{pac4}

Yo pana bhikkhu anupasampannaṁ padaso dhammaṁ vāceyya, pācittiyaṁ.

\pdfbookmark[3]{Pācittiya 5}{pac5}
\subsubsection*{\hyperref[exp5]{Pācittiya 5: Paṭhamasahaseyyasikkhāpadaṁ}}
\label{pac5}

Yo pana bhikkhu anupasampannena uttaridirattatirattaṁ sahaseyyaṁ kappeyya, pācittiyaṁ.

\pdfbookmark[3]{Pācittiya 6}{pac6}
\subsubsection*{\hyperref[exp6]{Pācittiya 6: Dutiyasahaseyyasikkhāpadaṁ}}
\label{pac6}

Yo pana bhikkhu mātugāmena sahaseyyaṁ kappeyya, pācittiyaṁ.

\pdfbookmark[3]{Pācittiya 7}{pac7}
\subsubsection*{\hyperref[exp7]{Pācittiya 7: Dhammadesanāsikkhāpadaṁ}}
\label{pac7}

Yo pana bhikkhu mātugāmassa uttarichappañcavācāhi dhammaṁ deseyya aññatra viññunā purisaviggahena, pācittiyaṁ.

\pdfbookmark[3]{Pācittiya 8}{pac8}
\subsubsection*{\hyperref[exp8]{Pācittiya 8: Bhūtārocanasikkhāpadaṁ}}
\label{pac8}

Yo pana bhikkhu anupasampannassa uttarimanussadhammaṁ āroceyya, bhūtasmiṁ pācittiyaṁ.

\pdfbookmark[3]{Pācittiya 9}{pac9}
\subsubsection*{\hyperref[exp9]{Pācittiya 9: Duṭṭhullārocanasikkhāpadaṁ}}
\label{pac9}

Yo pana bhikkhu bhikkhussa duṭṭhullaṁ āpattiṁ anupasampannassa āroceyya aññatra bhikkhusammutiyā, pācittiyaṁ.

\pdfbookmark[3]{Pācittiya 10}{pac10}
\subsubsection*{\hyperref[exp10]{Pācittiya 10: Paṭhavīkhaṇanasikkhāpadaṁ}}
\label{pac10}

Yo pana bhikkhu pathaviṁ khaṇeyya vā khaṇāpeyya vā pācittiyaṁ.

\begin{center}
  Musāvādavaggo paṭhamo
\end{center}

\subsection{Bhūtagāmavaggo}
\vspace{0.2cm}

\pdfbookmark[3]{Pācittiya 11}{pac11}
\subsubsection*{\hyperref[exp11]{Pācittiya 11: Bhūtagāmasikkhāpadaṁ}}
\label{pac11}

Bhūtagāmapātabyatāya pācittiyaṁ.

\pdfbookmark[3]{Pācittiya 12}{pac12}
\subsubsection*{\hyperref[exp12]{Pācittiya 12: Aññavādakasikkhāpadaṁ}}
\label{pac12}

Aññavādake, vihesake pācittiyaṁ.

\pdfbookmark[3]{Pācittiya 13}{pac13}
\subsubsection*{\hyperref[exp13]{Pācittiya 13: Ujjhāpanakasikkhāpadaṁ}}
\label{pac13}

Ujjhāpanake, khiyyanake pācittiyaṁ.

\pdfbookmark[3]{Pācittiya 14}{pac14}
\subsubsection*{\hyperref[exp14]{Pācittiya 14: Paṭhamasen'āsanasikkhāpadaṁ}}
\label{pac14}

Yo pana bhikkhu saṅghikaṁ mañcaṁ vā pīṭhaṁ vā bhisiṁ vā kocchaṁ vā ajjhokāse santharitvā vā santharāpetvā vā taṁ pakkamanto neva uddhareyya, na uddharāpeyya, anāpucchaṁ vā gaccheyya, pācittiyaṁ.

\pdfbookmark[3]{Pācittiya 15}{pac15}
\subsubsection*{\hyperref[exp15]{Pācittiya 15: Dutiyasen'āsanasikkhāpadaṁ}}
\label{pac15}

Yo pana bhikkhu saṅghike vihāre seyyaṁ santharitvā vā santharāpetvā vā taṁ pakkamanto neva uddhareyya, na uddharāpeyya, anāpucchaṁ vā gaccheyya, pācittiyaṁ.

\pdfbookmark[3]{Pācittiya 16}{pac16}
\subsubsection*{\hyperref[exp16]{Pācittiya 16: Anupakhajjasikkhāpadaṁ}}
\label{pac16}

Yo pana bhikkhu saṅghike vihāre jānaṁ pubbupagataṁ bhikkhuṁ anupakhajja seyyaṁ kappeyya “yassa sambādho bhavissati, so pakkamissatī”ti etadeva paccayaṁ karitvā anaññaṁ, pācittiyaṁ.

\pdfbookmark[3]{Pācittiya 17}{pac17}
\subsubsection*{\hyperref[exp17]{Pācittiya 17: Nikkaḍḍhanasikkhāpadaṁ}}
\label{pac17}

Yo pana bhikkhu bhikkhuṁ kupito anattamano saṅghikā vihārā nikkaḍḍheyya vā nikkaḍḍhāpeyya vā, pācittiyaṁ.

\pdfbookmark[3]{Pācittiya 18}{pac18}
\subsubsection*{\hyperref[exp18]{Pācittiya 18: Vehāsakuṭisikkhāpadaṁ}}
\label{pac18}

Yo pana bhikkhu saṅghike vihāre uparivehāsakuṭiyā āhaccapādakaṁ mañcaṁ vā pīṭhaṁ vā abhinisīdeyya vā abhinipajjeyya vā, pācittiyaṁ.

\pdfbookmark[3]{Pācittiya 19}{pac19}
\subsubsection*{\hyperref[exp19]{Pācittiya 19: Mahallakavihārasikkhāpadaṁ}}
\label{pac19}

Mahallakaṁ pana bhikkhunā vihāraṁ kārayamānena yāva dvārakosā aggaḷaṭṭhapanāya ālokasandhiparikammāya dvatticchadanassa pariyāyaṁ appaharite ṭhitena adhiṭṭhātabbaṁ, tato ce uttari appaharitepi ṭhito adhiṭṭhaheyya, pācittiyaṁ.

\pdfbookmark[3]{Pācittiya 20}{pac20}
\subsubsection*{\hyperref[exp20]{Pācittiya 20: Sappāṇakasikkhāpadaṁ}}
\label{pac20}

Yo pana bhikkhu jānaṁ sappāṇakaṁ udakaṁ tiṇaṁ vā mattikaṁ vā siñceyya vā siñcāpeyya vā, pācittiyaṁ.

\begin{center}
  Bhūtagāmavaggo dutiyo
\end{center}

\subsection{Bhikkhunovādavaggo}
\vspace{0.2cm}

\pdfbookmark[3]{Pācittiya 21}{pac21}
\subsubsection*{\hyperref[exp]{Pācittiya 21: Ovādasikkhāpadaṁ}}
\label{pac21}

Yo pana bhikkhu asammato bhikkhuniyo ovadeyya, pācittiyaṁ.

\pdfbookmark[3]{Pācittiya 22}{pac22}
\subsubsection*{\hyperref[exp22]{Pācittiya 22: Atthaṅgatasikkhāpadaṁ}}
\label{pac22}

Sammatopi ce bhikkhu atthaṅgate sūriye bhikkhuniyo ovadeyya, pācittiyaṁ.

\pdfbookmark[3]{Pācittiya 23}{pac23}
\subsubsection*{\hyperref[exp23]{Pācittiya 23: Bhikkhunupassayasikkhāpadaṁ}}
\label{pac23}

Yo pana bhikkhu bhikkhunupassayaṁ upasaṅkamitvā bhikkhuniyo ovadeyya aññatra samayā, pācittiyaṁ. Tatthāyaṁ samayo, gilānā hoti bhikkhunī, ayaṁ tattha samayo.

\pdfbookmark[3]{Pācittiya 24}{pac24}
\subsubsection*{\hyperref[exp24]{Pācittiya 24: Āmisasikkhāpadaṁ}}
\label{pac24}

Yo pana bhikkhu evaṁ vadeyya “āmisahetu therā bhikkhū bhikkhuniyo ovadantī”ti, pācittiyaṁ.

\pdfbookmark[3]{Pācittiya 25}{pac25}
\subsubsection*{\hyperref[exp25]{Pācittiya 25: Cīvaradānasikkhāpadaṁ}}
\label{pac25}

Yo pana bhikkhu aññātikāya bhikkhuniyā cīvaraṁ dadeyya aññatra pārivattakā, pācittiyaṁ.

\pdfbookmark[3]{Pācittiya 26}{pac26}
\subsubsection*{\hyperref[exp26]{Pācittiya 26: Cīvarasibbanasikkhāpadaṁ}}
\label{pac26}

Yo pana bhikkhu aññātikāya bhikkhuniyā cīvaraṁ sibbeyya vā sibbāpeyya vā, pācittiyaṁ.

\pdfbookmark[3]{Pācittiya 27}{pac27}
\subsubsection*{\hyperref[exp27]{Pācittiya 27: Saṁvidhānasikkhāpadaṁ}}
\label{pac27}

Yo pana bhikkhu bhikkhuniyā saddhiṁ saṁvidhāya ekaddhānamaggaṁ paṭipajjeyya antamaso gāmantarampi aññatra samayā, pācittiyaṁ. Tatthāyaṁ samayo, satthagamanīyo hoti maggo, sāsaṅkasammato, sappaṭibhayo, ayaṁ tattha samayo.

\pdfbookmark[3]{Pācittiya 28}{pac28}
\subsubsection*{\hyperref[exp28]{Pācittiya 28: Nāvābhiruhanasikkhāpadaṁ}}
\label{pac28}

Yo pana bhikkhu bhikkhuniyā saddhiṁ saṁvidhāya ekaṁ nāvaṁ abhiruheyya uddhaṅgāminiṁ vā adhogāminiṁ vā aññatra tiriyaṁ taraṇāya, pācittiyaṁ.

\pdfbookmark[3]{Pācittiya 29}{pac29}
\subsubsection*{\hyperref[exp29]{Pācittiya 29: Paripācitasikkhāpadaṁ}}
\label{pac29}

Yo pana bhikkhu jānaṁ bhikkhuniparipācitaṁ piṇḍapātaṁ bhuñjeyya aññatra pubbe gihisamārambhā, pācittiyaṁ.

\pdfbookmark[3]{Pācittiya 30}{pac30}
\subsubsection*{\hyperref[exp30]{Pācittiya 30: Rahonisajjasikkhāpadaṁ}}
\label{pac30}

Yo pana bhikkhu bhikkhuniyā saddhiṁ eko ekāya raho nisajjaṁ kappeyya, pācittiyaṁ.

\begin{center}
  Ovādavaggo tatiyo
\end{center}

\subsection{Bhojanavaggo}
\vspace{0.2cm}

\pdfbookmark[3]{Pācittiya 31}{pac31}
\subsubsection*{\hyperref[exp31]{Pācittiya 31: Āvasathapiṇḍasikkhāpadaṁ}}
\label{pac31}

Agilānena bhikkhunā eko āvasathapiṇḍo bhuñjitabbo. Tato ce uttari bhuñjeyya, pācittiyaṁ.

\pdfbookmark[3]{Pācittiya 32}{pac31}
\subsubsection*{\hyperref[exp32]{Pācittiya 32: Gaṇabhojanasikkhāpadaṁ}}
\label{pac32}

Gaṇabhojane aññatra samayā pācittiyaṁ. Tatthāyaṁ samayo, gilānasamayo, cīvaradānasamayo, cīvarakārasamayo, addhānagamanasamayo, nāvābhiruhanasamayo, mahāsamayo, samaṇabhattasamayo, ayaṁ tattha samayo.

\pdfbookmark[3]{Pācittiya 33}{pac33}
\subsubsection*{\hyperref[exp33]{Pācittiya 33: Paramparabhojanasikkhāpadaṁ}}
\label{pac33}

Paramparabhojane aññatra samayā pācittiyaṁ. Tatthāyaṁ samayo, gilānasamayo, cīvaradānasamayo, cīvarakārasamayo, ayaṁ tattha samayo.

\pdfbookmark[3]{Pācittiya 34}{pac34}
\subsubsection*{\hyperref[exp34]{Pācittiya 34: Kāṇamātusikkhāpadaṁ}}
\label{pac34}

Bhikkhuṁ paneva kulaṁ upagataṁ pūvehi vā manthehi vā abhihaṭṭhuṁ pavāreyya, ākaṅkhamānena bhikkhunā dvattipattapūrā paṭiggahetabbā. Tato ce uttari paṭiggaṇheyya, pācittiyaṁ. Dvattipattapūre paṭiggahetvā tato nīharitvā bhikkhūhi saddhiṁ saṁvibhajitabbaṁ, ayaṁ tattha sāmīci.

\pdfbookmark[3]{Pācittiya 35}{pac35}
\subsubsection*{\hyperref[exp35]{Pācittiya 35: Paṭhamapavāraṇāsikkhāpadaṁ}}
\label{pac35}

Yo pana bhikkhu bhuttāvī pavārito anatirittaṁ khādanīyaṁ vā bhojanīyaṁ vā khādeyya vā bhuñjeyya vā, pācittiyaṁ.

\pdfbookmark[3]{Pācittiya 36}{pac36}
\subsubsection*{\hyperref[exp36]{Pācittiya 36: Dutiyapavāraṇāsikkhāpadaṁ}}
\label{pac36}

Yo pana bhikkhu bhikkhuṁ bhuttāviṁ pavāritaṁ anatirittena khādanīyena vā bhojanīyena vā abhihaṭṭhuṁ pavāreyya “handa bhikkhu khāda vā bhuñja vā”ti jānaṁ āsādanāpekkho, bhuttasmiṁ pācittiyaṁ.

\pdfbookmark[3]{Pācittiya 37}{pac37}
\subsubsection*{\hyperref[exp37]{Pācittiya 37: Vikālabhojanasikkhāpadaṁ}}
\label{pac37}

Yo pana bhikkhu vikāle khādanīyaṁ vā bhojanīyaṁ vā khādeyya vā bhuñjeyya vā, pācittiyaṁ.

\pdfbookmark[3]{Pācittiya 38}{pac38}
\subsubsection*{\hyperref[exp38]{Pācittiya 38: Sannidhikārakasikkhāpadaṁ}}
\label{pac38}

Yo pana bhikkhu sannidhikārakaṁ khādanīyaṁ vā bhojanīyaṁ vā khādeyya vā bhuñjeyya vā, pācittiyaṁ.

\pdfbookmark[3]{Pācittiya 39}{pac39}
\subsubsection*{\hyperref[exp39]{Pācittiya 39: Paṇītabhojanasikkhāpadaṁ}}
\label{pac39}

Yāni kho pana tāni paṇītabhojanāni, seyyathidaṁ—sappi, navanītaṁ, telaṁ, madhu, phāṇitaṁ, maccho, maṁsaṁ, khīraṁ, dadhi. Yo pana bhikkhu evarūpāni paṇītabhojanāni agilāno attano atthāya viññāpetvā bhuñjeyya, pācittiyaṁ.

\pdfbookmark[3]{Pācittiya 40}{pac40}
\subsubsection*{\hyperref[exp40]{Pācittiya 40: Dantaponasikkhāpadaṁ}}
\label{pac40}

Yo pana bhikkhu adinnaṁ mukhadvāraṁ āhāraṁ āhareyya aññatra udakadantaponā, pācittiyaṁ.

\begin{center}
  Bhojanavaggo catuttho
\end{center}

\subsection{Acelakavaggo}
\vspace{0.2cm}

\pdfbookmark[3]{Pācittiya 41}{pac41}
\subsubsection*{\hyperref[exp41]{Pācittiya 41: Acelakasikkhāpadaṁ}}
\label{pac41}

Yo pana bhikkhu acelakassa vā paribbājakassa vā paribbājikāya vā sahatthā khādanīyaṁ vā bhojanīyaṁ vā dadeyya, pācittiyaṁ.

\pdfbookmark[3]{Pācittiya 42}{pac42}
\subsubsection*{\hyperref[exp42]{Pācittiya 42: Uyyojanasikkhāpadaṁ}}
\label{pac42}

Yo pana bhikkhu bhikkhuṁ “ehāvuso, gāmaṁ vā nigamaṁ vā piṇḍāya pavisissāmā”ti tassa dāpetvā vā adāpetvā vā uyyojeyya “gacchāvuso, na me tayā saddhiṁ kathā vā nisajjā vā phāsu hoti, ekakassa me kathā vā nisajjā vā phāsu hotī”ti etadeva paccayaṁ karitvā anaññaṁ, pācittiyaṁ.

\pdfbookmark[3]{Pācittiya 43}{pac43}
\subsubsection*{\hyperref[exp43]{Pācittiya 43: Sabhojanasikkhāpadaṁ}}
\label{pac43}

Yo pana bhikkhu sabhojane kule anupakhajja nisajjaṁ kappeyya, pācittiyaṁ.

\pdfbookmark[3]{Pācittiya 44}{pac44}
\subsubsection*{\hyperref[exp44]{Pācittiya 44: Rahopaṭicchannasikkhāpadaṁ}}
\label{pac44}

Yo pana bhikkhu mātugāmena saddhiṁ raho paṭicchanne āsane nisajjaṁ kappeyya, pācittiyaṁ.

\pdfbookmark[3]{Pācittiya 45}{pac45}
\subsubsection*{\hyperref[exp45]{Pācittiya 45: Rahonisajjasikkhāpadaṁ}}
\label{pac45}

Yo pana bhikkhu mātugāmena saddhiṁ eko ekāya raho nisajjaṁ kappeyya, pācittiyaṁ.

\pdfbookmark[3]{Pācittiya 46}{pac46}
\subsubsection*{\hyperref[exp46]{Pācittiya 46: Cārittasikkhāpadaṁ}}
\label{pac46}

Yo pana bhikkhu nimantito sabhatto samāno santaṁ bhikkhuṁ anāpucchā purebhattaṁ vā pacchābhattaṁ vā kulesu cārittaṁ āpajjeyya aññatra samayā, pācittiyaṁ. Tatthāyaṁ samayo, cīvaradānasamayo, cīvarakārasamayo, ayaṁ tattha samayo.

\pdfbookmark[3]{Pācittiya 47}{pac47}
\subsubsection*{\hyperref[exp47]{Pācittiya 47: Mahānāmasikkhāpadaṁ}}
\label{pac47}

Agilānena bhikkhunā catumāsappaccayapavāraṇā sāditabbā aññatra punapavāraṇāya, aññatra niccapavāraṇāya. Tato ce uttari sādiyeyya, pācittiyaṁ.

\pdfbookmark[3]{Pācittiya 48}{pac48}
\subsubsection*{\hyperref[exp48]{Pācittiya 48: Uyyuttasenāsikkhāpadaṁ}}
\label{pac48}

Yo pana bhikkhu uyyuttaṁ senaṁ dassanāya gaccheyya aññatra tathārūpappaccayā, pācittiyaṁ.

\pdfbookmark[3]{Pācittiya 49}{pac49}
\subsubsection*{\hyperref[exp49]{Pācittiya 49: Senāvāsasikkhāpadaṁ}}
\label{pac49}

Siyā ca tassa bhikkhuno kocideva paccayo senaṁ gamanāya, dirattatirattaṁ tena bhikkhunā senāya vasitabbaṁ. Tato ce uttari vaseyya, pācittiyaṁ.

\pdfbookmark[3]{Pācittiya 50}{pac50}
\subsubsection*{\hyperref[exp50]{Pācittiya 50: Uyyodhikasikkhāpadaṁ}}
\label{pac50}

Dirattatirattaṁ ce bhikkhu senāya vasamāno uyyodhikaṁ vā balaggaṁ vā senābyūhaṁ vā anīkadassanaṁ vā gaccheyya, pācittiyaṁ.

\begin{center}
  Acelakavaggo pañcamo
\end{center}

\subsection{Surāpānavaggo}
\vspace{0.2cm}

\pdfbookmark[3]{Pācittiya 51}{pac51}
\subsubsection*{\hyperref[exp51]{Pācittiya 51: Surāpānasikkhāpadaṁ}}
\label{pac51}

Surāmerayapāne pācittiyaṁ.

\pdfbookmark[3]{Pācittiya 52}{pac52}
\subsubsection*{\hyperref[exp52]{Pācittiya 52: Aṅgulipatodakasikkhāpadaṁ}}
\label{pac52}

Aṅgulipatodake pācittiyaṁ.

\pdfbookmark[3]{Pācittiya 53}{pac53}
\subsubsection*{\hyperref[exp53]{Pācittiya 53: Hassadhammasikkhāpadaṁ}}
\label{pac53}

Udake hasadhamme pācittiyaṁ.

\pdfbookmark[3]{Pācittiya 54}{pac54}
\subsubsection*{\hyperref[exp54]{Pācittiya 54: Anādariyasikkhāpadaṁ}}
\label{pac54}

Anādariye pācittiyaṁ.

\pdfbookmark[3]{Pācittiya 55}{pac55}
\subsubsection*{\hyperref[exp55]{Pācittiya 55: Bhiṁsāpanasikkhāpadaṁ}}
\label{pac55}

Yo pana bhikkhu bhikkhuṁ bhiṁsāpeyya, pācittiyaṁ.

\pdfbookmark[3]{Pācittiya 56}{pac56}
\subsubsection*{\hyperref[exp56]{Pācittiya 56: Jotikasikkhāpadaṁ}}
\label{pac56}

Yo pana bhikkhu agilāno visibbanāpekkho jotiṁ samādaheyya vā samādahāpeyya vā aññatra tathārūpappaccayā, pācittiyaṁ.

\pdfbookmark[3]{Pācittiya 57}{pac57}
\subsubsection*{\hyperref[exp57]{Pācittiya 57: Nahānasikkhāpadaṁ}}
\label{pac57}

Yo pana bhikkhu orenaddhamāsaṁ nahāyeyya aññatra samayā, pācittiyaṁ. Tatthāyaṁ samayo “diyaḍḍho māso seso gimhānan''ti “vassānassa paṭhamo māso” iccete aḍḍhateyyamāsā uṇhasamayo, pariḷāhasamayo, gilānasamayo, kammasamayo, addhānagamanasamayo, vātavuṭṭhisamayo, ayaṁ tattha samayo.

\pdfbookmark[3]{Pācittiya 58}{pac58}
\subsubsection*{\hyperref[exp58]{Pācittiya 58: Dubbaṇṇakaraṇasikkhāpadaṁ}}
\label{pac58}

Navaṁ pana bhikkhunā cīvaralābhena tiṇṇaṁ dubbaṇṇakaraṇānaṁ aññataraṁ dubbaṇṇakaraṇaṁ ādātabbaṁ nīlaṁ vā kaddamaṁ vā kāḷasāmaṁ vā. Anādā ce bhikkhu tiṇṇaṁ dubbaṇṇakaraṇānaṁ aññataraṁ dubbaṇṇakaraṇaṁ navaṁ cīvaraṁ paribhuñjeyya, pācittiyaṁ.

\pdfbookmark[3]{Pācittiya 59}{pac59}
\subsubsection*{\hyperref[exp59]{Pācittiya 59: Vikappanasikkhāpadaṁ}}
\label{pac59}

Yo pana bhikkhu bhikkhussa vā bhikkhuniyā vā sikkhamānāya vā sāmaṇerassa vā sāmaṇeriyā vā sāmaṁ cīvaraṁ vikappetvā appaccuddhāraṇaṁ paribhuñjeyya, pācittiyaṁ.

\pdfbookmark[3]{Pācittiya 60}{pac60}
\subsubsection*{\hyperref[exp60]{Pācittiya 60: Apanidhānasikkhāpadaṁ}}
\label{pac60}

Yo pana bhikkhu bhikkhussa pattaṁ vā cīvaraṁ vā nisīdanaṁ vā sūcigharaṁ vā kāyabandhanaṁ vā apanidheyya vā apanidhāpeyya vā antamaso hasāpekkhopi, pācittiyaṁ.

\begin{center}
  Surāpānavaggo chaṭṭho
\end{center}

\subsection{Sappāṇavaggo}
\vspace{0.2cm}

\pdfbookmark[3]{Pācittiya 61}{pac61}
\subsubsection*{\hyperref[exp61]{Pācittiya 61: Sañciccasikkhāpadaṁ}}
\label{pac61}

Yo pana bhikkhu sañcicca pāṇaṁ jīvitā voropeyya, pācittiyaṁ.

\pdfbookmark[3]{Pācittiya 62}{pac62}
\subsubsection*{\hyperref[exp62]{Pācittiya 62: Sappāṇakasikkhāpadaṁ}}
\label{pac62}

Yo pana bhikkhu jānaṁ sappāṇakaṁ udakaṁ paribhuñjeyya, pācittiyaṁ.

\pdfbookmark[3]{Pācittiya 63}{pac63}
\subsubsection*{\hyperref[exp63]{Pācittiya 63: Ukkoṭanasikkhāpadaṁ}}
\label{pac63}

Yo pana bhikkhu jānaṁ yathādhammaṁ nihatādhikaraṇaṁ punakammāya ukkoṭeyya, pācittiyaṁ.

\pdfbookmark[3]{Pācittiya 64}{pac64}
\subsubsection*{\hyperref[exp64]{Pācittiya 64: Duṭṭhullasikkhāpadaṁ}}
\label{pac64}

Yo pana bhikkhu bhikkhussa jānaṁ duṭṭhullaṁ āpattiṁ paṭicchādeyya, pācittiyaṁ.

\pdfbookmark[3]{Pācittiya 65}{pac65}
\subsubsection*{\hyperref[exp65]{Pācittiya 65: Ūnavīsativassasikkhāpadaṁ}}
\label{pac65}

Yo pana bhikkhu jānaṁ ūnavīsativassaṁ puggalaṁ upasampādeyya, so ca puggalo anupasampanno, te ca bhikkhū gārayhā, idaṁ tasmiṁ pācittiyaṁ.

\pdfbookmark[3]{Pācittiya 66}{pac66}
\subsubsection*{\hyperref[exp66]{Pācittiya 66: Theyyasatthasikkhāpadaṁ}}
\label{pac66}

Yo pana bhikkhu jānaṁ theyyasatthena saddhiṁ saṁvidhāya ekaddhānamaggaṁ paṭipajjeyya antamaso gāmantarampi, pācittiyaṁ.

\pdfbookmark[3]{Pācittiya 67}{pac67}
\subsubsection*{\hyperref[exp67]{Pācittiya 67: Saṁvidhānasikkhāpadaṁ}}
\label{pac67}

Yo pana bhikkhu mātugāmena saddhiṁ saṁvidhāya ekaddhānamaggaṁ paṭipajjeyya antamaso gāmantarampi, pācittiyaṁ.

\pdfbookmark[3]{Pācittiya 68}{pac68}
\subsubsection*{\hyperref[exp68]{Pācittiya 68: Ariṭṭhasikkhāpadaṁ}}
\label{pac68}

Yo pana bhikkhu evaṁ vadeyya “tathāhaṁ bhagavatā dhammaṁ desitaṁ ājānāmi, yathā yeme antarāyikā dhammā vuttā bhagavatā, te paṭisevato nālaṁ antarāyāyā”ti, so bhikkhu bhikkhūhi evamassa vacanīyo “māyasmā evaṁ avaca, mā bhagavantaṁ abbhācikkhi, na hi sādhu bhagavato abbhakkhānaṁ, na hi bhagavā evaṁ vadeyya, anekapariyāyenāvuso antarāyikā dhammā antarāyikā vuttā bhagavatā, alañca pana te paṭisevato antarāyāyā”ti. Evañca so bhikkhu bhikkhūhi vuccamāno tatheva paggaṇheyya, so bhikkhu bhikkhūhi yāvatatiyaṁ samanubhāsitabbo tassa paṭinissaggāya. Yāvatatiyañce samanubhāsiyamāno taṁ paṭinissajjeyya, iccetaṁ kusalaṁ. No ce paṭinissajjeyya, pācittiyaṁ.

\pdfbookmark[3]{Pācittiya 69}{pac69}
\subsubsection*{\hyperref[exp69]{Pācittiya 69: Ukkhittasambhogasikkhāpadaṁ}}
\label{pac69}

Yo pana bhikkhu jānaṁ tathāvādinā bhikkhunā akaṭānudhammena taṁ diṭṭhiṁ appaṭinissaṭṭhena saddhiṁ sambhuñjeyya vā, saṁvaseyya vā, saha vā seyyaṁ kappeyya, pācittiyaṁ.

\pdfbookmark[3]{Pācittiya 70}{pac70}
\subsubsection*{\hyperref[exp70]{Pācittiya 70: Kaṇṭakasikkhāpadaṁ}}
\label{pac70}

Samaṇuddesopi ce evaṁ vadeyya “tathāhaṁ bhagavatā dhammaṁ desitaṁ ājānāmi, yathā yeme antarāyikā dhammā vuttā bhagavatā, te paṭisevato nālaṁ antarāyāyā”ti, so samaṇuddeso bhikkhūhi evamassa vacanīyo “māvuso, samaṇuddesa evaṁ avaca, mā bhagavantaṁ abbhācikkhi, na hi sādhu bhagavato abbhakkhānaṁ, na hi bhagavā evaṁ vadeyya, anekapariyāyenāvuso, samaṇuddesa antarāyikā dhammā antarāyikā vuttā bhagavatā, alañca pana te paṭisevato antarāyāyā”ti, evañca so samaṇuddeso bhikkhūhi vuccamāno tatheva paggaṇheyya, so samaṇuddeso bhikkhūhi evamassa vacanīyo “ajjatagge te, āvuso, samaṇuddesa na ceva so bhagavā satthā apadisitabbo, yampi caññe samaṇuddesā labhanti bhikkhūhi saddhiṁ dirattatirattaṁ sahaseyyaṁ, sāpi te natthi, cara pire, vinassā”ti. Yo pana bhikkhu jānaṁ tathānāsitaṁ samaṇuddesaṁ upalāpeyya vā, upaṭṭhāpeyya vā, sambhuñjeyya vā, saha vā seyyaṁ kappeyya, pācittiyaṁ.

\begin{center}
  Sappāṇakavaggo sattamo
\end{center}

\subsection{Sahadhammikavaggo}
\vspace{0.2cm}

\pdfbookmark[3]{Pācittiya 71}{pac71}
\subsubsection*{\hyperref[exp71]{Pācittiya 71: Sahadhammikasikkhāpadaṁ}}
\label{pac71}

Yo pana bhikkhu bhikkhūhi sahadhammikaṁ vuccamāno evaṁ vadeyya “na tāvāhaṁ, āvuso, etasmiṁ sikkhāpade sikkhissāmi, yāva na aññaṁ bhikkhuṁ byattaṁ vinayadharaṁ paripucchāmī”ti, pācittiyaṁ. Sikkhamānena, bhikkhave, bhikkhunā aññātabbaṁ paripucchitabbaṁ paripañhitabbaṁ, ayaṁ tattha sāmīci.

\pdfbookmark[3]{Pācittiya 72}{pac72}
\subsubsection*{\hyperref[exp72]{Pācittiya 72: Vilekhanasikkhāpadaṁ}}
\label{pac72}

Yo pana bhikkhu pātimokkhe uddissamāne evaṁ vadeyya “kiṁ panimehi khuddānukhuddakehi sikkhāpadehi uddiṭṭhehi, yāvadeva kukkuccāya vihesāya vilekhāya saṁvattantī”ti, sikkhāpadavivaṇṇake pācittiyaṁ.

\pdfbookmark[3]{Pācittiya 73}{pac73}
\subsubsection*{\hyperref[exp73]{Pācittiya 73: Mohanasikkhāpadaṁ}}
\label{pac73}

Yo pana bhikkhu anvaddhamāsaṁ pātimokkhe uddissamāne evaṁ vadeyya “idāneva kho ahaṁ jānāmi, ayampi kira dhammo suttāgato suttapariyāpanno anvaddhamāsaṁ uddesaṁ āgacchatī”ti. Tañce bhikkhuṁ aññe bhikkhū jāneyyuṁ nisinnapubbaṁ iminā bhikkhunā dvattikkhattuṁ pātimokkhe uddissamāne, ko pana vādo bhiyyo, na ca tassa bhikkhuno aññāṇakena mutti atthi, yañca tattha āpattiṁ āpanno, tañca yathādhammo kāretabbo, uttari cassa moho āropetabbo “tassa te, āvuso, alābhā, tassa te dulladdhaṁ, yaṁ tvaṁ pātimokkhe uddissamānena sādhukaṁ aṭṭhiṁ katvā manasi karosī”ti, idaṁ tasmiṁ mohanake pācittiyaṁ.

\pdfbookmark[3]{Pācittiya 74}{pac74}
\subsubsection*{\hyperref[exp74]{Pācittiya 74: Pahārasikkhāpadaṁ}}
\label{pac74}

Yo pana bhikkhu bhikkhussa kupito anattamano pahāraṁ dadeyya, pācittiyaṁ.

\pdfbookmark[3]{Pācittiya 75}{pac75}
\subsubsection*{\hyperref[exp75]{Pācittiya 75: Talasattikasikkhāpadaṁ}}
\label{pac75}

Yo pana bhikkhu bhikkhussa kupito anattamano talasattikaṁ uggireyya, pācittiyaṁ.

\pdfbookmark[3]{Pācittiya 76}{pac76}
\subsubsection*{\hyperref[exp76]{Pācittiya 76: Amūlakasikkhāpadaṁ}}
\label{pac76}

Yo pana bhikkhu bhikkhuṁ amūlakena saṅghādisesena anuddhaṁseyya, pācittiyaṁ.

\pdfbookmark[3]{Pācittiya 77}{pac77}
\subsubsection*{\hyperref[exp77]{Pācittiya 77: Sañciccasikkhāpadaṁ}}
\label{pac77}

Yo pana bhikkhu bhikkhussa sañcicca kukkuccaṁ upadaheyya “itissa muhuttampi aphāsu bhavissatī”ti etadeva paccayaṁ karitvā anaññaṁ, pācittiyaṁ.

\pdfbookmark[3]{Pācittiya 78}{pac78}
\subsubsection*{\hyperref[exp78]{Pācittiya 78: Upassutisikkhāpadaṁ}}
\label{pac78}

Yo pana bhikkhu bhikkhūnaṁ bhaṇḍanajātānaṁ kalahajātānaṁ vivādāpannānaṁ upassutiṁ tiṭṭheyya “yaṁ ime bhaṇissanti, taṁ sossāmī”ti etadeva paccayaṁ karitvā anaññaṁ, pācittiyaṁ.

\pdfbookmark[3]{Pācittiya 79}{pac79}
\subsubsection*{\hyperref[exp79]{Pācittiya 79: Kammappaṭibāhanasikkhāpadaṁ}}
\label{pac79}

Yo pana bhikkhu dhammikānaṁ kammānaṁ chandaṁ datvā pacchā khīyanadhammaṁ āpajjeyya, pācittiyaṁ.

\pdfbookmark[3]{Pācittiya 80}{pac80}
\subsubsection*{\hyperref[exp80]{Pācittiya 80: Chandaṁ-adatvā-gamanasikkhāpadaṁ}}
\label{pac80}

Yo pana bhikkhu saṅghe vinicchayakathāya vattamānāya chandaṁ adatvā uṭṭhāyāsanā pakkameyya, pācittiyaṁ.

\pdfbookmark[3]{Pācittiya 81}{pac81}
\subsubsection*{\hyperref[exp81]{Pācittiya 81: Dubbalasikkhāpadaṁ}}
\label{pac81}

Yo pana bhikkhu samaggena saṅghena cīvaraṁ datvā pacchā khīyanadhammaṁ āpajjeyya “yathāsanthutaṁ bhikkhū saṅghikaṁ lābhaṁ pariṇāmentī”ti, pācittiyaṁ.

\pdfbookmark[3]{Pācittiya 82}{pac82}
\subsubsection*{\hyperref[exp82]{Pācittiya 82: Pariṇāmanasikkhāpadaṁ}}
\label{pac82}

Yo pana bhikkhu jānaṁ saṅghikaṁ lābhaṁ pariṇataṁ puggalassa pariṇāmeyya, pācittiyaṁ.

\begin{center}
  Sahadhammikavaggo aṭṭhamo
\end{center}

\subsection{Rājavaggo}
\vspace{0.2cm}

\pdfbookmark[3]{Pācittiya 83}{pac83}
\subsubsection*{\hyperref[exp83]{Pācittiya 83: Antepurasikkhāpadaṁ}}
\label{pac83}

Yo pana bhikkhu rañño khattiyassa muddhābhisittassa anikkhantarājake aniggataratanake pubbe appaṭisaṁvidito indakhīlaṁ atikkāmeyya, pācittiyaṁ.

\pdfbookmark[3]{Pācittiya 84}{pac84}
\subsubsection*{\hyperref[exp84]{Pācittiya 84: Ratanasikkhāpadaṁ}}
\label{pac84}

Yo pana bhikkhu ratanaṁ vā ratanasammataṁ vā aññatra ajjhārāmā vā ajjhāvasathā vā uggaṇheyya vā uggaṇhāpeyya vā, pācittiyaṁ. Ratanaṁ vā pana bhikkhunā ratanasammataṁ vā ajjhārāme vā ajjhāvasathe vā uggahetvā vā uggahāpetvā vā nikkhipitabbaṁ “yassa bhavissati, so harissatī”ti, ayaṁ tattha sāmīci.

\pdfbookmark[3]{Pācittiya 85}{pac85}
\subsubsection*{\hyperref[exp85]{Pācittiya 85: Vikālagāmappavesanasikkhāpadaṁ}}
\label{pac85}

Yo pana bhikkhu santaṁ bhikkhuṁ anāpucchāvikāle gāmaṁ paviseyya aññatra tathārūpā accāyikā karaṇīyā, pācittiyaṁ.

\pdfbookmark[3]{Pācittiya 86}{pac86}
\subsubsection*{\hyperref[exp86]{Pācittiya 86: Sūcigharasikkhāpadaṁ}}
\label{pac86}

Yo pana bhikkhu aṭṭhimayaṁ vā dantamayaṁ vā visāṇamayaṁ vā sūcigharaṁ kārāpeyya, bhedanakaṁ pācittiyaṁ.

\pdfbookmark[3]{Pācittiya 87}{pac87}
\subsubsection*{\hyperref[exp87]{Pācittiya 87: Mañcapīṭhasikkhāpadaṁ}}
\label{pac87}

Navaṁ pana bhikkhunā mañcaṁ vā pīṭhaṁ vā kārayamānena aṭṭhaṅgulapādakaṁ kāretabbaṁ sugataṅgulena aññatra heṭṭhimāya aṭaniyā. Taṁ atikkāmayato chedanakaṁ pācittiyaṁ.

\pdfbookmark[3]{Pācittiya 88}{pac88}
\subsubsection*{\hyperref[exp88]{Pācittiya 88: Tūlonaddhasikkhāpadaṁ}}
\label{pac88}

Yo pana bhikkhu mañcaṁ vā pīṭhaṁ vā tūlonaddhaṁ kārāpeyya, uddālanakaṁ pācittiyaṁ.

\pdfbookmark[3]{Pācittiya 89}{pac89}
\subsubsection*{\hyperref[exp89]{Pācittiya 89: Nisīdanasikkhāpadaṁ}}
\label{pac89}

Nisīdanaṁ pana bhikkhunā kārayamānena pamāṇikaṁ kāretabbaṁ, tatridaṁ pamāṇaṁ, dīghaso dve vidatthiyo sugatavidatthiyā, tiriyaṁ diyaḍḍhaṁ, dasā vidatthi. Taṁ atikkāmayato chedanakaṁ pācittiyaṁ.

\pdfbookmark[3]{Pācittiya 90}{pac90}
\subsubsection*{\hyperref[exp90]{Pācittiya 90: Kaṇḍuppaṭicchādisikkhāpadaṁ}}
\label{pac90}

Kaṇḍuppaṭicchādiṁ pana bhikkhunā kārayamānena pamāṇikā kāretabbā, tatridaṁ pamāṇaṁ, dīghaso catasso vidatthiyo sugatavidatthiyā, tiriyaṁ dve vidatthiyo. Taṁ atikkāmayato chedanakaṁ pācittiyaṁ.

\pdfbookmark[3]{Pācittiya 91}{pac91}
\subsubsection*{\hyperref[exp91]{Pācittiya 91: Vassikasāṭikasikkhāpadaṁ}}
\label{pac91}

Vassikasāṭikaṁ pana bhikkhunā kārayamānena pamāṇikā kāretabbā, tatridaṁ pamāṇaṁ, dīghaso cha vidatthiyo sugatavidatthiyā, tiriyaṁ aḍḍhateyyā. Taṁ atikkāmayato chedanakaṁ pācittiyaṁ.

\pdfbookmark[3]{Pācittiya 92}{pac92}
\subsubsection*{\hyperref[exp92]{Pācittiya 92: Nandasikkhāpadaṁ}}
\label{pac92}
Yo pana bhikkhu sugatacīvarappamāṇaṁ cīvaraṁ kārāpeyya, atirekaṁ vā, chedanakaṁ pācittiyaṁ. Tatridaṁ sugatassa sugatacīvarappamāṇaṁ, dīghaso nava vidatthiyo sugatavidatthiyā, tiriyaṁ cha vidatthiyo, idaṁ sugatassa sugatacīvarapamāṇanti.

\begin{center}
  Ratanavaggo navamo
\end{center}

\medskip

\begin{center}
Uddiṭṭhā kho āyasmanto dvenavuti pācittiyā dhammā.

\smallskip

Tatth'āyasmante pucchāmi: Kacci'ttha parisuddhā?\\
Dutiyam'pi pucchāmi: Kacci'ttha parisuddhā?\\
Tatiyam'pi pucchāmi: Kacci'ttha parisuddhā?

\smallskip

Parisuddh'etth'āyasmanto, tasmā tuṇhī, evam'etaṁ dhārayāmi.
\end{center}

\begin{outro}
  Pācittiyā niṭṭhitā
\end{outro}

\clearpage
